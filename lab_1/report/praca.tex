\documentclass[10pt,english, openany]{book}

%%%%%%%%%%%%%%%%%%%%%%%%%%%%%% Loading packages that alter the style
\usepackage[]{graphicx}
\usepackage[]{color}
\usepackage{alltt}
\usepackage[T1]{fontenc}
\usepackage[utf8]{inputenc}
\setcounter{secnumdepth}{3}
\setcounter{tocdepth}{3}
\setlength{\parskip}{\smallskipamount}
\setlength{\parindent}{0pt}

% Set page margins
\usepackage[top=100pt,bottom=100pt,left=68pt,right=66pt]{geometry}

% Package used for placeholder text
\usepackage{lipsum}

% Prevents LaTeX from filling out a page to the bottom
\raggedbottom

% Adding both languages
\usepackage[english, italian]{babel}

% All page numbers positioned at the bottom of the page
\usepackage{fancyhdr}
\fancyhf{} % clear all header and footers
\fancyfoot[C]{\thepage}
\renewcommand{\headrulewidth}{0pt} % remove the header rule
\pagestyle{fancy}

% Changes the style of chapter headings
\usepackage{titlesec}
\titleformat{\chapter}
   {\normalfont\LARGE\bfseries}{\thechapter.}{1em}{}
% Change distance between chapter header and text
\titlespacing{\chapter}{0pt}{50pt}{2\baselineskip}

% Adds table captions above the table per default
\usepackage{float}
\floatstyle{plaintop}
\restylefloat{table}

% Adds space between caption and table
\usepackage[tableposition=top]{caption}

% Adds hyperlinks to references and ToC
\usepackage{hyperref}
\hypersetup{hidelinks,linkcolor = black} % Changes the link color to black and hides the hideous red border that usually is created

% If multiple images are to be added, a folder (path) with all the images can be added here 
\graphicspath{ {Figures/} }

% Separates the first part of the report/thesis in Roman numerals
\frontmatter
%--------------dla kodu i justyfikacji
\usepackage{listings}
\usepackage{ragged2e}
%-------------------------------

%%%%%%%%%%%%%%%%%%%%%%%%%%%%%% Starts the document
\begin{document}

%%% Selects the language to be used for the first couple of pages
\selectlanguage{english}

%%%%% Adds the title page
\begin{titlepage}
	\clearpage\thispagestyle{empty}
	\centering
	\vspace{0.5cm}    
    \centering \includegraphics[scale=0.5]{agh.jpg}    
	\vspace{0.5cm}   
	
		\vspace{2cm}
	{\huge \textbf{Metody programowania równoległego}} \\
	\vspace{0.5cm} 
	{\Huge \textbf{Sprawozdanie z mierzenia opóźnienia i przepustowości w klastrze}} \\
	\vspace{6cm}	
	{\Large autor: Ilona Tomkowicz\\}
	\vspace{0.5cm} 
	{\Large Akademia Górniczo-Hutnicza\\
	Wydział Informatyki, Elektroniki i Telekomunikacji,\\
	Informatyka, II stopień, I semestr \par}
	   \vspace{0.5cm} 
	% Set the date
	{\Large 08 marca 2020 \par}
	
	\pagebreak

\end{titlepage}

% Adds a table of contents
\tableofcontents{}

%%%%%%%%%%%%%%%%%%%%%%%%%%%%%%%%%%%%%%%%%%%%%%%%%%%%%%%%%%%%%%%%%%%%%%%%%%%%%%%%%%%%%%%%%%%%
%%%%%%%%%%%%%%%%%%%%%%%%%%%%%%%%%%%%%%%%%%%%%%%%%%%%%%%%%%%%%%%%%%%%%%%%%%%%%%%%%%%%%%%%%%%%
%%%%% Text body starts here!
\mainmatter
%1. kodów programów
%2. danych pomiarowych
%3. pomiarów opóźnienia i wykresów przepustowości dla różnych parametrów / konfiguracji - może być w osobnych plikach lub złożone w formie dokumentu, np. PDF. Wykresy mają mieć opisane osie oraz informacje o konfiguracji w jakiej test został uruchomiony (jakie maszynki, jakie parametry etc).
%4. wniosków do wykresów

\chapter{Kody źródłowe}\label{chapt:kod}
\section{Komunikacja z użyciem Send i Recv}
\lstinputlisting[language=C]{codes/hw.c}

\section{Komunikacja z użyciem Isend i Irecv}
\lstinputlisting[language=C]{codes/hw_b.c}

\section{Komunikacja z użyciem jednego węzła i pamięci współdzielonej}
Konfiguracja pliku allnodes zawierała tylko adres tego węzła, a program uruchamiano na tym samym węźle.
\lstinputlisting[language=C]{codes/hosts}

\section{Komunikacja z użyciem jednego węzła i połączenia sieciowego}
Konfiguracja pliku allnodes zawierała tylko adres tego węzła, a program uruchamiano na innym węźle, w tym wypadku na 02.

\section{Komunikacja z użyciem dwóch węzłów, będących fizycznie na tej samej maszynie i połączenia sieciowego}
Konfiguracja pliku allnodes zawierała adresy używanych węzłów (01, 03), a program uruchamiano na węźle 02.
\lstinputlisting[language=C]{codes/hosts_2}

\section{Komunikacja z użyciem dwóch węzłów, będących fizycznie na różnych maszynach i połączenia sieciowego}
Konfiguracja pliku allnodes zawierała adresy używanych węzłów (05, 06), a program uruchamiano na węźle 02.
\lstinputlisting[language=C]{codes/hosts_3}

\chapter{Dane pomiarowe}

Pomiary wykonano dla paczek od 64 B do wielkości maksymalnej, która nie blokowała wykonania programu. Zrobiono 10000 iteracji w połączeniu ping-pong w celu zmierzenia poniższych parametrów. Dokładne dane jakich maszyn użyto w kolejnych konfiguracjach znajdują się w poprzednim rozdziale.

\begin{table}[H]
\caption{Jeden węzeł, pamięć współdzielona}
\begin{center}
\begin{tabular}{|c|c|}
\hline
rozmiar {[}bit{]} & Send/recv przepustowość {[}Mbit/s{]} \\ \hline
64              & 347.872                              \\ \hline
256             & 1328.89                              \\ \hline
1024            & 4148.32                              \\ \hline
4096            & 12235.5                              \\ \hline
16384           & 38741.4                              \\ \hline
65536           & 60750.5                              \\ \hline
262144          & 61720.4                              \\ \hline
1048576         & 51421.7                              \\ \hline
\end{tabular}
\end{center}
\end{table}

\begin{table}[H]
\caption{Jeden węzeł, pamięć współdzielona}
\begin{center}
\begin{tabular}{|c|c|}
\hline
size {[}bit{]} & Isend/Irecv przepustowość {[}Mbit/s{]} \\ \hline
64           & 75.871                                 \\ \hline
256          & 322.474                                \\ \hline
1024         & 1240.93                                \\ \hline
4096         & 7566.22                                \\ \hline
16384        & 27147.9                                \\ \hline
65536        & 81429.6                                \\ \hline
262144       & 119653                                 \\ \hline
1048576      & 55264.7                                \\ \hline
\end{tabular}
\end{center}
\end{table}


\begin{table}[H]
\caption{Jeden węzeł, połączenie sieciowe}
\begin{center}
\begin{tabular}{|c|c|}
\hline
size {[}bit{]} & Send/recv przepustowość {[}Mbit/s{]} \\ \hline
64           & 339.148                              \\ \hline
256          & 1354.45                              \\ \hline
1024         & 4119.87                              \\ \hline
4096         & 13120.9                              \\ \hline
16384        & 36961.9                              \\ \hline
65536        & 46702.7                              \\ \hline
262144       & 54996.8                              \\ \hline
\end{tabular}
\end{center}
\end{table}

\begin{table}[H]
\caption{Jeden węzeł, połączenie sieciowe}
\begin{center}
\begin{tabular}{|c|c|}
\hline
size {[}bit{]} & Isend/Irecv przepustowość {[}Mbit/s{]} \\ \hline
64           & 74.3444                                \\ \hline
256          & 275.714                                \\ \hline
1024         & 1601.85                                \\ \hline
4096         & 4002.86                                \\ \hline
16384        & 25237.1                                \\ \hline
65536        & 71933.1                                \\ \hline
262144       & 75605.2                                \\ \hline
1048576      & 43859.2                                \\ \hline
\end{tabular}
\end{center}
\end{table}

\begin{table}[H]
\caption{Dwa węzły, jeden host}
\begin{center}
\begin{tabular}{|c|c|}
\hline
size {[}bit{]} & Send/recv przepustowość {[}Mbit/s{]} \\ \hline
64           & 319.243                              \\ \hline
256          & 1156.55                              \\ \hline
1024         & 4638.69                              \\ \hline
4096         & 12530.4                              \\ \hline
16384        & 31242.5                              \\ \hline
65536        & 59722.3                              \\ \hline
262144       & 53988.5                              \\ \hline
\end{tabular}
\end{center}
\end{table}


\begin{table}[H]
\caption{Dwa węzły, jeden host}
\begin{center}
\begin{tabular}{|c|c|}
\hline
size {[}bit{]} & Isend/Irecv przepustowość {[}Mbit/s{]} \\ \hline
64           & 74.64                                  \\ \hline
256          & 337.766                                \\ \hline
1024         & 1003.36                                \\ \hline
4096         & 4407.19                                \\ \hline
16384        & 26874.5                                \\ \hline
65536        & 60076.7                                \\ \hline
262144       & 120085                                 \\ \hline
1048576      & 36403.5                                \\ \hline
\end{tabular}
\end{center}
\end{table}

\begin{table}[H]
\caption{Dwa węzły, różne hosty}
\begin{center}
\begin{tabular}{|c|c|}
\hline
size {[}bit{]} & Send/recv przepustowość {[}Mbit/s{]} \\ \hline
64           & 3.09821                              \\ \hline
256          & 12.1787                              \\ \hline
1024         & 49.11                                \\ \hline
4096         & 76.0858                              \\ \hline
16384        & 120.212                              \\ \hline
65536        & 440.153                              \\ \hline
262144       & 1206.13                              \\ \hline
524288       & 1917.71                              \\ \hline
\end{tabular}
\end{center}
\end{table}

\begin{table}[H]
\caption{Dwa węzły, różne hosty}
\begin{center}
\begin{tabular}{|c|c|}
\hline
size {[}bit{]} & Isend/Irecv przepustowość {[}Mbit/s{]} \\ \hline
64           & 9.89011                                \\ \hline
256          & 36.2111                                \\ \hline
1024         & 92.1024                                \\ \hline
4096         & 360.693                                \\ \hline
16384        & 895.855                                \\ \hline
65536        & 2388.71                                \\ \hline
262144       & 2862.97                                \\ \hline
1048576      & 3153.6                                 \\ \hline
\end{tabular}
\end{center}
\end{table}

\begin{table}[H]
\caption{Porównanie opóźnienia dla małej paczki 64 bitów}
\begin{center}
\begin{tabular}{|c|c|l|}
\hline
       & Send+Recv       & Isend+Irecv     \\ \hline
zad\_1 & 1.94335e-07 sec & 7.73275e-07 sec \\ \hline
zad\_2 & 2.21586e-07 sec & 8.59272e-07 sec \\ \hline
zad\_3 & 1.99199e-07 sec & 8.57234e-07 sec \\ \hline
zad\_4 & 2.06115e-05 sec & 6.07103e-06 sec \\ \hline
\end{tabular}
\end{center}
\end{table}

\chapter{Wykresy dla różnych konfiguracji}

\centering \includegraphics[scale=0.65]{pics/z1.png}    
\centering \includegraphics[scale=0.63]{pics/z2.png}    
\centering \includegraphics[scale=0.63]{pics/z3.png}    
\centering \includegraphics[scale=0.6]{pics/z4.png}    

\chapter{Wnioski}
\justifying
Zgodnie z oczekiwaniami największą przepustowość ma połączenie lokalne bez użycia sieci. Biorąc pod uwagę najlepszy przypadek (połączenie asynchroniczne przesyłające 32768 B - pik czerwonej linii), przepustowość dochodzi aż do 125000Mbit/s. Mimo, że połączenie przez sieć przez dwa węzły (01,03) daje podobne rezultaty w maksimum, to krzywa wykresu jest bardziej stroma. 

Nie każdy test udało się nasycić dojściem do maksymalnej przepustowości. W większości dla komunikacji asynchronicznej można zaobserwować maksymalny pik, a potem spadek przepustowości. Dla komunikacji synchronicznej także istnieje niższa granica maksymalnych paczek, bo 32768 B to zwykle największa ilość jaką udało się przesłać. Dla komunikacji asynchronicznej ta wielkość dochodzi do 131072 B.

W każdej konfiguracji komunikacja asynchroniczna ma większą przepustowość, jednak opóźnienie związane z tą komunikacją jest większe. Jest to różnica mniejsza niż jeden rząd wielkości, więc dla używanych danych nie ma dużego znaczenia.

Opóźnienie pomiędzy konfiguracjami dla tych samych operacji jest porównywalne dla wszystkich przypadków poza ostatnim. W konfiguracji, gdzie łączymy dwa różne hosty opóźnienie rośnie nawet o 2 rzędy wielkości, co jest odczuwalną różnicą. Wniosek z tego jest taki, że dopóki węzły są fizycznie na tym samym hoście to nie będzie miało drastycznego znaczenia czy użyjemy do obliczeń połączenia sieciowego, pamięci współdzielonej, komunikacji synchronicznej czy asynchronicznej. Dopiero przy komunikacji dwóch różnych fizycznie hostów takie szczegóły konfiguracyjne dadzą się odczuć.


\end{document}